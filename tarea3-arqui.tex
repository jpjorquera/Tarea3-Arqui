\documentclass[11pt,letterpaper]{article}
\usepackage[top=2.0cm, bottom=3cm, left=2.0cm, right=2.0cm]{geometry}
\usepackage[utf8]{inputenc}
\usepackage[T1]{fontenc}
\usepackage[spanish]{varioref}
\usepackage[activeacute, spanish, es-tabla]{babel}
\usepackage{fancyhdr}
\usepackage{multicol}
\usepackage{float}
\usepackage{textcomp}
\usepackage{ae,aecompl}
\usepackage{amssymb,amsmath}
\usepackage[pdftex]{graphicx}
\usepackage{askmaps}
\usepackage{multirow}
\pagestyle{fancy} 
\pagenumbering{arabic} 
\renewcommand{\headrulewidth}{0pt} 
\setlength{\headsep}{20pt} 
\setlength{\headheight}{65pt} 
\setlength{\textheight}{600pt} 
\setlength{\columnsep}{15pt} 
\newcommand{\universidad}{\small{Universidad Técnica Federico Santa María}}
\newcommand{\campus}{\small{Campus Santiago San Joaquín}}

% Definiciones de Título e Integrantes de Experiencia
\newcommand{\titulo}{Informe Tarea 2 \\ Arquitectura y Organización de Computadores}
\newcommand{\integrantes}{\begin{tabular}{c}
Juan Pablo Jorquera  201573533-6 \\
Cristian Navarrete 201573549-2\\
\end{tabular}}

\renewcommand{\maketitle}
{
\thispagestyle{fancy}
\begin{center}
\begin{Large}
\textbf{\titulo}\\
\end{Large}
\end{center}
\vspace{0.3cm}
}


%ENCABEZADO

\fancyhead[R]{\begin{minipage}[b]{0.405\textwidth}
\begin{center}
\universidad \\ 
\campus \\ 
\integrantes
\end{center}
\end{minipage}}
\fancyhead[L]{\vspace{15pt}\includegraphics[height=1.6cm]{Escudo.png}}
%%%%%%%%%%%%%%%%%%%%%%%%%%%%%%%%%%%%%%%%%%%%%%%%
%                                              %
% AQUI TERMINAN LAS DEFINICIONES DE ENCABEZADO %
% Y EMPIEZA EL CUERPO DEL DOCUMENTO            %
%                                              %
%%%%%%%%%%%%%%%%%%%%%%%%%%%%%%%%%%%%%%%%%%%%%%%%

\begin{document}
\setcounter{secnumdepth}{0}
\maketitle
\section{Circuito}
Se busca realizar el circuito con las características requeridas, para ello en primer lugar realizamos el diagrama de estados correspondiente.

Se agregó en la etiqueta la parte de las secuencias que se va cumpliendo en cada estado (dígitos finales) para tener una mejor visualización.
\begin{center}
\includegraphics[width=15cm]{diag-estados.png}
\end{center}

Luego se enumeró arbitrariamente cada estado, como se muestra a continuación:
\vspace{0.2cm}
\begin{table}[h]
\centering
\caption{Codificación de Estados}
\vspace{0.2cm}
\begin{tabular}{c|c|c|c|}
\cline{2-4}
                             & \multicolumn{3}{c|}{Dígito} \\ \hline
\multicolumn{1}{|c|}{Estado} & A       & B       & C       \\ \hline
\multicolumn{1}{|c|}{$q_0$}     & 0       & 0       & 0       \\ \hline
\multicolumn{1}{|c|}{$q_1$}     & 0       & 0       & 1       \\ \hline
\multicolumn{1}{|c|}{$q_2$}     & 0       & 1       & 0       \\ \hline
\multicolumn{1}{|c|}{$q_3$}     & 0       & 1       & 1       \\ \hline
\multicolumn{1}{|c|}{$q_4$}     & 1       & 0       & 0       \\ \hline
\multicolumn{1}{|c|}{$q_5$}     & 1       & 0       & 1       \\ \hline
\end{tabular}
\end{table}

\newpage
Luego, con la enumeración de estados, se generó  la tabla de transición y de salida con la que se va a trabajar más adelante.
\begin{table}[h]
\centering
\caption{Tabla de Transición y Salida}
\vspace{0.2cm}
\label{tab_output}
\begin{tabular}{|l|c|c|c|c||l|c|c|c|c|}
\hline
\multicolumn{4}{|c|}{Estado}   & Input & \multicolumn{4}{l|}{Siguiente Estado} & Output \\ \hline
No.                & A & B & C & X      & No.       & D       & E      & F      &  Y      \\ \hline
\multirow{2}{*}{0} & 0 & 0 & 0 & 0     & 0         & 0       & 0      & 0      & 0      \\ \cline{2-10} 
                   & 0 & 0 & 0 & 1     & 1         & 0       & 0      & 1      & 0      \\ \hline
\multirow{2}{*}{1} & 0 & 0 & 1 & 0     & 2         & 0       & 1      & 0      & 0      \\ \cline{2-10} 
                   & 0 & 0 & 1 & 1     & 1         & 0       & 0      & 1      & 0      \\ \hline
\multirow{2}{*}{2} & 0 & 1 & 0 & 0     & 0         & 0       & 0      & 0      & 0      \\ \cline{2-10} 
                   & 0 & 1 & 0 & 1     & 3         & 0       & 1      & 1      & 0      \\ \hline
\multirow{2}{*}{3} & 0 & 1 & 1 & 0     & 4         & 1       & 0      & 0      & 1      \\ \cline{2-10} 
                   & 0 & 1 & 1 & 1     & 5         & 1       & 0      & 1      & 1      \\ \hline
\multirow{2}{*}{4} & 1 & 0 & 0 & 0     & 0         & 0       & 0      & 0      & 0      \\ \cline{2-10} 
                   & 1 & 0 & 0 & 1     & 3         & 0       & 1      & 1      & 0      \\ \hline
\multirow{2}{*}{5} & 1 & 0 & 1 & 0     & 2         & 0       & 1      & 0      & 0      \\ \cline{2-10} 
                   & 1 & 0 & 1 & 1     & 1         & 0       & 0      & 1      & 0      \\ \hline
\multirow{2}{*}{}  & 1 & 1 & 0 & 0     &           & X       & X      & X      & X      \\ \cline{2-10} 
                   & 1 & 1 & 0 & 1     &           & X       & X      & X      & X      \\ \hline
\multirow{2}{*}{}  & 1 & 1 & 1 & 0     &           & X       & X      & X      & X      \\ \cline{2-10} 
                   & 1 & 1 & 1 & 1     &           & X       & X      & X      & X      \\ \hline
\end{tabular}
\end{table}

A continuación se realizan los mapas de Karnaugh para cada salida correspondiente, usando la notación incluida en la tabla anterior.

\vspace{0.2cm}
\begin{center}
\askmap{4}{$D$}{{$A$}{$B$}{$C$}{$X$}}%
{000000110000XXXX}%
{%
\put(2,1){\oval(1.9,1.9)}
}
\askmap{4}{$E$}{{$A$}{$B$}{$C$}{$X$}}%
{001001000110XXXX}%
{%
\put(2,2.5){\oval(1.9,0.8)}
\put(3,2.5){\oval(1.9,0.7)}
\put(4,0.5){\oval(1.9,0.8)[l]}
\put(0,0.5){\oval(1.9,0.8)[r]}
}

\askmap{4}{$F$}{{$A$}{$B$}{$C$}{$X$}}%
{010101010101XXXX}%
{%
\put(2,2){\oval(3.9,1.8)}
}\askmap{4}{$Y$}{{$A$}{$B$}{$C$}{$X$}}%
{000000110000XXXX}%
{%
\put(2,1){\oval(1.8,1.8)}
}
\end{center}
\vspace{0.2cm}

Obteniendo así las siguientes funciones:
\begin{itemize}
	\item{$D = BC$}
	\item{$E = \overline{B}C\overline{X} + A\overline{C}X + B\overline{C}X$}
	\item{$F = X$}
	\item{$Y  = BC$}
\end{itemize}

Finalmente se construye el circuito con las funciones obtenidas, usando flip-flop tipo D donde corresponda.
\vspace{0.2cm}

\section{Módulo HDL}
Para ello se realizaron tres funciones, una que hace el flip-flop d y dos para cada función lógica compuesta donde simplemente se obtienen las variables $D$, $E$ y la salida $Y$, según las funciones obtenidas anteriormente. Luego, haciendo uso de dichas funciones, se genera el circuito completo, haciendo uso las variables faltantes como lo son la entrada $X$ y el dígito $F$. Cabe destacar que al acualizar los parámetros, también se actualizan las variables que dependen de ellas, por lo que no es necesario realizar iteraciones, ya que con los flip-flops se realizan los cambios necesarios.

\newpage
\section{Extra}
Para realizar esto hacemos un análisis de la tabla de verdad del flip flop jk conocida, que se incluye a continuación:
\begin{table}[h]
\centering
\caption{Tabla de Verdad flip-flop JK}
\vspace{0.2cm}
\label{tab-jk}
\begin{tabular}{|c|c|c|c|}
\hline
J & K & Estado & Estado Siguiente \\ \hline
0 & 0 & 0      & 0                \\ \hline
0 & 0 & 1      & 1                \\ \hline
0 & 1 & X      & 0                \\ \hline
1 & 0 & X      & 1                \\ \hline
1 & 1 & 0      & 1                \\ \hline
1 & 1 & 1      & 0                \\ \hline
\end{tabular}
\end{table}
\vspace{0.2cm}
Para reemplazar el flip-flop d buscamos los casos en que el estado se mantenga constante, por lo que inmediatamente cuando j y k son uno no nos sirve, luego vemos que si j y k son ambos cero, la salida depende del estado actual, el que no hay manera de saber exactamente el valor que toma en que momento como para poder invertirlo y no afectar la entrada jk en la siguiente iteración, por los que solo nos queda el caso en que ambos son distintos, donde nos damos cuenta que el estado siguiente es equivalente al de j, por lo que corresponde a la entrada y salida del flip flop d, entonces finalmente se podría reemplazar haciendo uso de un circuito de la forma:
\begin{center}
\includegraphics[width=2cm]{jk.png}
\end{center}

Es por esto que el circuito-jk que se hizo para reemplazar el anterior es simplemente agregando este tipo de puertas en el lugar de flip flop d.


\end{document}
