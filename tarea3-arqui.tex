\documentclass[11pt,letterpaper]{article}
\usepackage[top=2.0cm, bottom=3cm, left=2.0cm, right=2.0cm]{geometry}
\usepackage[utf8]{inputenc}
\usepackage[T1]{fontenc}
\usepackage[spanish]{varioref}
\usepackage[activeacute, spanish, es-tabla]{babel}
\usepackage{fancyhdr}
\usepackage{multicol}
\usepackage{float}
\usepackage{textcomp}
\usepackage{ae,aecompl}
\usepackage{amssymb,amsmath}
\usepackage[pdftex]{graphicx}
\usepackage{askmaps}
\usepackage{multirow}
\usepackage{hyperref}
\pagestyle{fancy} 
\pagenumbering{arabic} 
\renewcommand{\headrulewidth}{0pt} 
\setlength{\headsep}{20pt} 
\setlength{\headheight}{65pt} 
\setlength{\textheight}{600pt} 
\setlength{\columnsep}{15pt}
\setlength{\parskip}{1em}
\newcommand{\universidad}{\small{Universidad Técnica Federico Santa María}}
\newcommand{\campus}{\small{Campus Santiago San Joaquín}}

% Definiciones de Título e Integrantes de Experiencia
\newcommand{\titulo}{Informe Tarea 3 \\ Arquitectura y Organización de Computadores}
\newcommand{\integrantes}{\begin{tabular}{c}
Juan Pablo Jorquera  201573533-6 \\
Cristian Navarrete 201573549-2\\
\end{tabular}}

\renewcommand{\maketitle}
{
\thispagestyle{fancy}
\begin{center}
\begin{Large}
\textbf{\titulo}\\
\end{Large}
\end{center}
\vspace{0.3cm}
}


%ENCABEZADO

\fancyhead[R]{\begin{minipage}[b]{0.405\textwidth}
\begin{center}
\universidad \\ 
\campus \\ 
\integrantes
\end{center}
\end{minipage}}
\fancyhead[L]{\vspace{15pt}\includegraphics[height=1.6cm]{Escudo.png}}
%%%%%%%%%%%%%%%%%%%%%%%%%%%%%%%%%%%%%%%%%%%%%%%%
%                                              %
% AQUI TERMINAN LAS DEFINICIONES DE ENCABEZADO %
% Y EMPIEZA EL CUERPO DEL DOCUMENTO            %
%                                              %
%%%%%%%%%%%%%%%%%%%%%%%%%%%%%%%%%%%%%%%%%%%%%%%%

\begin{document}
\setcounter{secnumdepth}{0}
\maketitle
\section{MCD}
En primer lugar se dejó al principio en .data espacio para modificar inputs para probar máximos común divisores. El problema en sí se resolvió usando el \href{https://es.wikipedia.org/wiki/Algoritmo_de_Euclides#Algoritmo_de_Euclides_tradicional}{Algoritmo de Euclides}, para ello en primer lugar se identificó el menor de ambos números (almacenados en $a_0$ y $a_1$), intercambiándolos de ser necesario, de modo que el menor de ellos quede en $a_0$. 

Luego se creó un loop $mcd$ que va buscando el resto de la división de ambos para aplicar el Algoritmo, en caso de ser $0$ el resto significa que terminó el proceso; de no ser así, se vuelve a iterar usando como nuevos valores el menor de los antes divididos y el resto calculado.

No es necesario hacer caso base, ya que en caso de no encontrar, simplemente el resto llegará a $1$, el cual debe entregar como resto $0$.

Al ser $0$ el resto, el $beq$ dentro del loop lo dirige hacia el final, donde para facilitar la visualización se imprime el valor encontrado (que también queda en $t_0$) y se almacena en la salida creada al principio. 

\section{Lucas}


\section{Fibonacci}
Para comenzar se creó el label enésimo para recibir el término de la serie que se quiere calcular, el cual almacenamos en $a_0$. 

Para continuar, definimos los dos primeros términos en $s_0$ y $s_1$, respectivamente y se realizan saltos para entregar el resultado en caso de que esos sean los términos que se busquen.

Luego se utilizó $t_0$ como un contador, el cual usaremos para verificar si se llegó al término enésimo y $t_1$ como auxiliar para ir calculando la suma de los términos actuales, para poder así actualizar $s_0$ y $s_1$ a los siguientes elementos de la serie.

Finalmente al finalizar el loop se imprime el valor, el cual también queda en $s_0$ y se guarda en la salida definida al principio.

Cabe destacar que como el MIPS trabaja en 32 bits y para enteros, utiliza el primer bit para el signo, el máximo término que puede calcular es cuando $n=46$, ya que es el último que cabe en $2^{31}-1$.

\section{Factorial}


\section{Extra}


\end{document}
